% vim: spelllang=en_gb

\documentclass[12pt,a4paper,draft]{scrartcl}
\usepackage{ifdraft}

% --------------------
% Set Language Options
% --------------------

\usepackage[nswissgerman,french,main=english]{babel}
\usepackage[autostyle,english=american,german=swiss]{csquotes}
\MakeOuterQuote{"}

\usepackage[shortcuts]{extdash}

% --------------
% Font & Symbols
% --------------

\usepackage[warnings-off={mathtools-colon,mathtools-overbracket}]{unicode-math}
\usepackage[oldstyle,proportional]{libertinus}

% ---------------
% Set Page Layout
% ---------------

% Get length of 65 characters
%\setlxvchars

\usepackage[driver=auto]{geometry}
% A5: 148mm × 210mm
% A4: 210mm × 297mm
\geometry{
  width=140mm,
  height=217mm,
  marginparsep=3mm,
  marginparwidth=30mm,
}
\ifdraft{\geometry{
  inner=10mm,
  marginparwidth=50mm
}}{}


% ---------------------
% Load Various Packages
% ---------------------

% Various Math Environments
\usepackage{mathtools}
\usepackage{amsthm,thmtools}
\usepackage{physics} % various shortcuts

% Bibliography
\usepackage{biblatex}
\addbibresource{bibliography.bib}

% For general figures
\usepackage{graphicx}
\usepackage{tikz}
\usetikzlibrary{babel,cd}
\tikzcdset{arrow style=math font}

% For lists
\usepackage[shortlabels]{enumitem}

% For better Tables
\usepackage{tabularray}

% For more fine grained typesetting
\ifdraft{}{\usepackage{microtype}}

% Links and stuff
\usepackage[final]{hyperref}
\usepackage[noabbrev]{cleveref}

% For Todonotes
\usepackage[obeyDraft]{luatodonotes}

% --------------------------------------------
% Define Theorem Environments & Math Operators
% --------------------------------------------

\declaretheorem[numberwithin=section]{theorem}
\declaretheorem[sibling=theorem]{lemma, proposition, corollary}
\declaretheorem[sibling=theorem,style=definition]{definition, example}
\declaretheorem[sibling=theorem,style=remark]{remark}

\DeclareMathOperator{\im}{im}
\DeclareMathOperator{\Aut}{Aut}
\DeclareMathOperator{\Diff}{Diff}
\DeclareMathOperator{\GL}{GL}
\DeclareMathOperator{\HF}{HF}
\DeclareMathOperator{\HM}{HM}
\DeclareMathOperator{\Hom}{Hom}
\DeclareMathOperator{\Ext}{Ext}
\DeclareMathOperator{\Tor}{Tor}

\begin{document}
\title{Exotic Tori from ATFs oder so}
\author{JoJoJo}

\maketitle

\section{Introduction}

\begin{definition}
  Let $k ∈ ℕ$ such that $0<k≤d$ and $a ∈ (0,∞)$. Through nodal slides we can arrange the ATF on $B_{dpq}$ such that the line $x₂=a$ intersects the branch cut line between the $(k-1)$-th and $k$-th degenerated fibre. $T_k(a)$ is defined to be the fibre over the intersection point of these two lines.
\end{definition}

\begin{theorem}
    \label{thm:bdpqexotic}
  Let $U ⊂ H¹(T_k(a),ℝ) ∖ \{\text{branch cut line}\}$.
  The restriction of the displacement energy germ to $U$ is given by
  \[ \eval{S_{T_k(a)}^e}_U (x,y) = a+\max\{x,x(1-kpq)-kp²y\}\todo{so oder so ähnlich…} \]
\end{theorem}


Let \(d,p,q ∈ ℕ\) such that \(d≥\) and \(p,q\) coprime with \(1≤q<p\) or \(q=0,p=1\), and \(0<a_1<…<a_d\) real integers.
Let \(P\) be the polynomial \(P(z) = \prod_{i=1}^d (z^p-a_i)\).
Define the manifold \(M_P\) by
\[M_P = \qty{(z_1,z_2,z_3) ∈ ℂ³ \mid z₁z₂ + P(z₃)=0 } \; .\]
We define the Hamiltonian system
\[\symbf{H}(z_1,z_2,z_3) = \qty(\abs{z_3}^2, \frac{1}{2}\qty(\abs{z_1}^2-\abs{z_2}^2))\]

Let \(μ_p\) be the group of \(p\)-Th roots of unity acting on \(M_P\) by
\[μ \cdot (z₁,z₂,z₃) = \qty(μz₁,μ^{-1}z₂,μ^q z₃), \quad μ ∈ μ_p \; .\]
This is a free  action, so we can define the quotient \(B_{dpq} = M_P/μ_p\). The Hamiltonian system \(\symbf{H}\) is invariant under the action, so it descends to a Hamiltonian system on \(B_{dpq}\).

\section{Upper bound on displacement energy: Probes}

\section{Short interlude: Homology of \texorpdfstring{$B_{dpq}$}{Bdpq}}
\label{sec:homology}

In order to calculate the lower bound for the displacement energy of a torus \(L(x,y)\)\todo{definieren}, we will need to calculate a basis for \(H_2(B_{dpq},L(x,y)\).

\(B_{dpq}\) deformation retracts\todo{Vielleicht den deformation retract kurz kommentieren} to the preimage of the branch cut line segment shown in \cref{fig:branch_cut_retraction}. 
This can be understood as follows: If there were no critical points on the line, this would be a solid torus \(T = S¹×D²\).
We pick \((1,0),(0,1) ∈ H₁(∂T)\) to be the classes generated by \(S¹×\text{pt},\text{pt}×∂D²\) respectively.
At each critical point we collapse a loop along homology class \((-q,p)\).
Up to homotopy this is the same as attaching a disk along \((-q,p)\). \todo{Warum?} \todo{In der ganzen Diskussion könnte man meinen dass der Torus $T(x,y)$ vor dem ersten Kreuz liegen muss damit es stimmt?\\
– Ah guter punkt. Aber \(T(x,y)\) sollte \emph{nach} allen kreuzen liegen dass es stimmt.}
Again up to homotopy we can also require that the \(d\) discs \(D_1,…,D_d\) are attached along \(∂T\).
Let us call this space $S$.

\begin{figure}
  \centering
  \missingfigure{retaction to branch cut}
  \caption{asdf}
  \label{fig:branch_cut_retraction}
\end{figure}

Let us look at the long exact sequence of homology for the pair \((B_{dpq},L(x,y))\). This pair is homotopy equivalent to \((S,∂T)\).

\[
\begin{tikzcd}
  H_2(∂T) \ar[r,"0"] &
  H_2(S) \ar[r,hook]\ar[d,"≅"] &
  H_2(S,∂T) \ar[r]\ar[d,"≅"] &
  H_1(∂T) \ar[r]\ar[d,"≅"] &
  H_1(S) \ar[d,"≅"]
  \\
  &
  ℤ^{d-1} &
  ℤ^{d+1} &
  ℤ² &
  ℤ_p
\end{tikzcd}
\]

The first horizontal map is zero since \(∂T\) retracts to a circle in \(S\).
Homology \(H_2(S)\) can be seen as follows: By contracting the solid torus \(T\) in \(S\) to a circle, we see that \(S\) is homotopic to a circle with \(d\) discs glued to its boundary by a degree \(p\) map.
So \(H_2(S)\) is generated by spheres \(\qty{S_2,…,S_d}\), \(S_k = D_1-D_{k+1}\).
\(H_2(S,∂T)\) is generated by the discs \(D_0 = \text{pt}×D²,D₁,…,D_d\). In \(B_{dpq}\), these discs can be seen, where the disc intersecting the toric boundary collapses the \((0,1)\) cycle in the toric fibre \(L(x,y)\) and the discs intersecting the critical points collapse the \((q,-p)\) cycle (see \cref{fig:homology_generating_discs}).
The elements $S_2,\ldots,S_d \in H_2$ can be realized by embedded Lagragian spheres fibering over the segments between the nodes in the ATF -- these are so-called \emph{visible Lagrangians}, see \cite[section 7.4]{evans2021atfs}.
The last boundary map $\partial \colon H_2(S,∂T) \rightarrow H_1(\partial T)$ is given by $\partial D_0 = (0,1),\, \partial D_i = (-q,p)$, meaning that the last horizontal map maps $(1,0)$ to the generator of $ℤ_p$.


\begin{figure}
  \centering
  \missingfigure{Generating discs}
  \caption{asdf}
  \label{fig:homology_generating_discs}
\end{figure}

\section{Lower Bound on Displacement Energy: Minimal J-holomorphic Curves}

Let \(L(x,y)\) a fibre torus, where \((x,y)\) is not over the branch cut line. In \cite{chekanov1998} the following is proven:

\begin{theorem}
  \label{thm:chekanov}
  Let \(L ⊂ X\) be a Lagrangian submanifold \todo{Compact Lagrangian in a tame symplectic manifold. Let $J$ be...}. Then the displacement energy satisfies
  \[e(L) ≥ \min{σ_D(X,J),σ_S(X,J)}\] \todo{$\sigma_D(X,L,J)$}
\end{theorem}

\subsection{Displacement energy and exotic tori in \texorpdfstring{\(B_{dpq}\)}{Bdpq}}

After a mutation, the ATF base diagram of $B_{dpq}$ is the right half-plane decorated with $d$ nodes on the ray in the direction $(p,q)$. The position of the nodes can be varied by nodal slides. In this subsection we denote by $L(x_0,y_0)$ the almost toric fibre over the point $(x_0,y_0)$ with $x > 0$. Fibres $L(\lambda p,\lambda q)$ for $\lambda > 0$ are monotone whenever they are not on a node. We compute displacement energy of the non-monotone fibres. Note that $L(x_0,y_0)$ yields a well-defined torus up to Hamiltonian isotopy, i.e.\ it is independent of the nodal slides, see \todo{referenz}. 

\begin{proposition}
\label{prop:bdpq}
Let $L(x_0,y_0)$ be a non-monotone ATF-fibre of $B_{dpq}$. Then it has displacement energy
\[e(B_{dpq}, L(x_0,y_0)) = x_0 .\]
\end{proposition}

\begin{proof}
    The upper bound follows from probes. For the lower bound, we use \cref{thm:chekanov} together with \cref{lem:bdpqdisk}. 
\end{proof}

\begin{lemma}
\label{lem:bdpqdisk}
    There is an $\omega$-compatible almost complex structre $J$ such that 
    \[\min \{\sigma_D(B_{dpq}, L(x_0,y_0), J) , \sigma_S(B_{dpq}, J)\} = x_0. \]  
\end{lemma}

\begin{proof}
    Independently of the choice of $J$, there are no $J$-holomorphic spheres in $B_{dpq}$, since $H_2(B_{dpq})$ has admits a basis the elements of which can be realized by Lagrangian spheres. In order to deal with disks we choose a particular $J$ as follows. Perform nodal slides such that all nodes lie in $\{x > x_0 + \varepsilon \}$. The subset $U = \pi^{-1}(\{x < x_0 + \varepsilon\})$ is symplectomorphic to $D(x_0 + \varepsilon) \times T^*S^1$. Take the standard complex structure coming from this symplectomorphism on $U$ and extend it to an arbitrary almost complex structure $J$ taming $\omega$. Since the projection 
    \[ p \colon (U, J\vert_U) \rightarrow (D(x_0 + \varepsilon), i) \]
    is holomorphic, the maximum principle\todo{gibts hierfür noch eine referenz? Weil als armen Studenten würde mich sowas aufregen. Ich bin vom Wort allein noch nicht ganz überzeugt…} implies that the image of any $J$-disk with boundary on $T(x_0,y_0)$ is contained in $U$. Since $H_2(U,T(x_0,y_0))$ is generated by the disk $D_0$ (see \cref{sec:homology}), and since there is an obvious $J$-disk $u$ with $[u] = D_0$, the claim follows.
\end{proof}

\todo{Add proof of \cref{thm:bdpqexotic}.}



\subsection{\texorpdfstring{\(D_0\)}{D₀} stays a Minimal J-Disks for Suitable Embeddings}

Let \((n,a)\) be two coprime integers, \(μ_n\) the group of \(n\)-th roots of unity. Let \(μ_n\) act on \(ℂ^2\) by \(μ(z₁,z₂) = ( μ z_1,μ^a z_2)\).
Let \(A(n,a) = ℂ^2/μ_n\) be the quotient space.
This space is an orbifold, with one orbifold point at \([(0,0)]\).

We define the Hamiltonian system on \(A(n,a)\) by 
\begin{equation}
  \label{eqn:hamsysAna}
  \symbf{G}(z_1,z_2) = \frac{1}{2}\qty(\abs{z_2}², \frac{1}{n}\qty(\abs{z₁}²+a\abs{z₂}²)) \;.
\end{equation}

With this Hamiltonian system the moment polytope is a wedge with edges pointing along vectors \((1,0), (n,a)\), as seen in \cref{fig:Ana_moment_polytope}. \(A(n,a)\) has a almost complex structure \(J\) coming from the canonical complex structure on \(ℂ²\).

\begin{figure}
  \centering
  \missingfigure{\(A(n,a)\) moment polytope}
  \caption{Moment polytope of \(A(n,a)\) with given by Hamiltonian system \(G\)}
  \label{fig:Ana_moment_polytope}
\end{figure}

Let \(B(a) ∈ ℂ^n\) be the open ball in \(ℂ^n\) of radius \(\sqrt{\frac{a}{π}}\).
In \cite[appendix~A]{chekanovschlenk2015} the following lemma is proven:
\begin{lemma}
  \label{lem:hyperannulus}
  Let \(a_+ > a_- ≥ 0\).
  Let \(u: Σ → B(a_+)∖ \overline{B(a_-)}\) be a J-holomorphic curve such that the closure of \(u(Σ)\) in \(ℂ^n\)intersects \(∂B(a_-)\).
  Then \(∫_u ω ≥ a_+ - a_-\).
\end{lemma}

We give the slight generalization:

\begin{lemma}
  \label{lem:hyperannulus2}
  Let \(a_+ > a_- > 0\), and
  \[X = \symbf{G}^{-1}\qty(\qty{c₁(0,1)+c₂(n,a) \mid a_- < c₁ + c₂ < a_+}) ⊂ A(n,a) \;,\]
  equipped with the almost complex structure of \(A(n,a)\).

  Let \(u: Σ → X\) be a J-holomorphic curve whose closure intersects
  \[\symbf{H}^{-1}\qty(\qty{c₁(0,1)+c₂(n,a) \mid a_- = c₁ + c₂}) \; .\]

  Then \(∫_u ω ≥ a_+ - a_-\).
\end{lemma}

\begin{remark}
  \label{rem:hyperannulus3}
  Suppose we have a moment polytope \(Δ\) of a (almost) toric symplectic manifold or orbifold \(\symbf{H} : M → Δ\) with two non-parallel edges given by the two primitive vectors $u₁,u₂$, as in \cref{fig:cutting_out_a_hyperannulens}.
  Suppose without loss of generality that the edges intersect in the origin.
  Then the subset
  \[X = \symbf{H}^{-1}\qty(\qty{c₁u₁+c₂u₂ \mid a_- < c₁ + c₂ < a_+})\]
  with \(a_±\) such that \(a_± u₁, a_± u₂ ∈ Δ\), can be transformed by a \(T ∈ \GL(ℤ²)\), such that \(Tu₁=(0,1), Tu₂=(n,a)\), for some coprime integers \(n,a\).

  With this transformation we can view \(X\) as a subset of \(A(n,a)\). Equipping \(M\) with an extension of the almost complex structure coming from \(A(n,a)\), we get that J-curves in \(M\) intersecting
  \[\symbf{H}^{-1}\qty(\qty{c₁u₁+c₂u₂ \mid a_- = c₁ + c₂})\]
  must have at least area \(a_+ - a_-\).
\end{remark}

\begin{figure}
  \centering
  \missingfigure{Hyperannulens inside a moment polytope.}
  \caption{Hyperannulens \todo{\textbf{NOOOOOOOOOOOO!}} inside a moment polytope.}
  \label{fig:cutting_out_a_hyperannulens}
\end{figure}
\begin{proof}
  Since the action of \(μ_n\) is free in \((ℂ^*)²\), the projection map \(π : (ℂ^*)² → A(n,a) ∖ \{[(0,0)]\}\) is an n-fold covering map.

  As shown below, the preimage \(π^{-1}(X)\) is \(B(na_+) ∖ \overline{B(na_-)}\), and \(u\) lifts to a J-curve \(\tilde{u}\), i.e. a curve making the diagram
  \[
    \begin{tikzcd}
      Σ' \ar[r,"\tilde{u}"] \ar[d,"\tilde{π}"] & ℂ² \ar[d,"π"] \\
      Σ \ar[r,"u"] & A(n,a)
    \end{tikzcd}
  \]
  commute, where \(\tilde{π} : Σ' → Σ\) is some n-fold covering of \(Σ\).

  Using \cref{lem:hyperannulus}, we get that the symplectic area of \(\tilde{u}\) is at least \(n(a_+ - a_-)\), and since \(\tilde{u}\) is an n-fold covering of \(u\), \(u\) has at least symplectic area \(a_+-a_-\), as desired.

  To show that \(π^{-1}(X) = B(na_+) ∖ \overline{B(na_-)}\), note that the Hamiltonian system \(\symbf{G}\) from \cref{eqn:hamsysAna}, is given by a linear transformation of the standard system on \(ℂ²\)
  \[\symbf{H}(z₁,z₂) = \frac{1}{2}\qty(\abs{z₁}²,\abs{z₂}²) \;,\]
  given by the matrix
  \[\mqty(0 & 1 \\ \frac{1}{n} & \frac{a}{n})\todo{Das ist jetzt nicht wie bei Evans linkswirkend, sondern normal…} \;,\]
  whose inverse maps \(\symbf{G}(X)\) to \(\symbf{H}(B(na_+) ∖ \overline{B(na_-)})\), and since \(π\) maps fibres to fibres, the claim follows.
\end{proof}

Define \(B_{dpq}(a) := \symbf{G}\todo{was ist \(G\) genau?}^{-1}\qty{c₁(0,1)+c₂(dp²,dpq-1) \mid c₁+c₂ < a)}\)\todo{gibts davon noch eine schönere definition?}, and let \((X,ω,J)\) be a symplectic manifold with tame almost complex structure.

Suppose we have an embedding \(B_{dqp}(2a) → X\), and that all J-spheres in \(X\) have at least area \(a\). Using \cref{rem:hyperannulus3,lem:hyperannulus2}, we get that the minimal J-curve with boundary on the torus \(T_k(a)\) has area \(a\), and since there are no smaller J-spheres by assumption, using \cref{thm:chekanov}, we get that the displacement energy of \(T_k(a)\) is at least \(a\).\todo{Brauchen wir hier noch ein ε platz?}

\iffalse
\begin{proof}
  Let \(r ∈ (a_-,a_+)\) such that the intersection \(u(Σ) ∩ ∂B_r\) is transversal.
  Then this intersection is an immersed 1-dimensional manifold, so it is a collection of immersed circles.
  Let \(γ\) be a parametrization of one of these circles.
  We choose a local holomorphic reparametrization of \(u\) as follows: 
  \begin{align*}
    \tilde{u} : S¹xI &→ ℂ^n \\
    u(t,0) &= γ(t)
  \end{align*} \todo{ja, was jetzt?}
  
  Then
  \begin{align*}
    F'(r) &= ∫_{u(Σ) ∩ ∂B_r} u^* ω \\
          &≥ ∫_{S¹} ω(\pdv{u∘s}{t},\pdv{u∘s}{r}) \dd t \\
          &= ∫_{S¹} ω(\pdv{u}{t}+\pdv{s}{t}\pdv{u}{s},\pdv{s}{r}\pdv{u}{s}) \dd t \\
          &= ∫_{S¹} \pdv{s}{r} ω(\pdv{u}{t},i \pdv{u}{t}) \dd t \\
          &= ∫_{S¹} \pdv{s}{r} \abs{\dot{γ}(t)}^2 \dd t \\
          &≥ l^2(γ) ∫_{S¹} \pdv{r}{s} \dd t
  \end{align*}

  Also
  \begin{align*}
    l^2(γ) &≥ ∫_{S¹} γ^* α_n = ∫_{S¹}⟨\dot{γ},ξ⟩ \dd t
  \end{align*}
\end{proof}
\fi



\printbibliography

\end{document}
