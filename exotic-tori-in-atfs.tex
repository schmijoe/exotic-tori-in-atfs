% vim: spelllang=en_gb

\documentclass[12pt,a4paper,draft]{scrartcl}
\usepackage{ifdraft}

% --------------------
% Set Language Options
% --------------------

\usepackage[nswissgerman,french,main=english]{babel}
\usepackage[autostyle,english=american,german=swiss]{csquotes}
\MakeOuterQuote{"}

\usepackage[shortcuts]{extdash}

% --------------
% Font & Symbols
% --------------

\usepackage[warnings-off={mathtools-colon,mathtools-overbracket}]{unicode-math}
\usepackage[oldstyle,proportional]{libertinus}

% ---------------
% Set Page Layout
% ---------------

% Get length of 65 characters
%\setlxvchars

\usepackage[driver=auto]{geometry}
% A5: 148mm × 210mm
% A4: 210mm × 297mm
\geometry{
  width=140mm,
  height=217mm,
  marginparsep=3mm,
  marginparwidth=30mm,
}
\ifdraft{\geometry{
  inner=10mm,
  marginparwidth=50mm
}}{}


% ---------------------
% Load Various Packages
% ---------------------

% Various Math Environments
\usepackage{mathtools}
\usepackage{amsthm,thmtools}
\usepackage{physics} % various shortcuts

% Bibliography
\usepackage{biblatex}
\addbibresource{bibliography.bib}

% For general figures
\usepackage{graphicx}
\usepackage{tikz}
\usetikzlibrary{babel,cd}
\tikzcdset{arrow style=math font}

% For lists
\usepackage[shortlabels]{enumitem}

% For better Tables
\usepackage{tabularray}

% For more fine grained typesetting
\ifdraft{}{\usepackage{microtype}}

% Links and stuff
\ifdraft{}{\usepackage{hyperref}}
\usepackage[noabbrev]{cleveref}

% For Todonotes
\usepackage[obeyDraft]{luatodonotes}

% --------------------------------------------
% Define Theorem Environments & Math Operators
% --------------------------------------------

\declaretheorem[numberwithin=section]{theorem}
\declaretheorem[sibling=theorem]{lemma, proposition, corollary}
\declaretheorem[sibling=theorem,style=definition]{definition, example}
\declaretheorem[sibling=theorem,style=remark]{remark}

\DeclareMathOperator{\im}{im}
\DeclareMathOperator{\Aut}{Aut}
\DeclareMathOperator{\Diff}{Diff}
\DeclareMathOperator{\GL}{GL}
\DeclareMathOperator{\HF}{HF}
\DeclareMathOperator{\HM}{HM}
\DeclareMathOperator{\Hom}{Hom}
\DeclareMathOperator{\Ext}{Ext}
\DeclareMathOperator{\Tor}{Tor}

\begin{document}
\title{Exotic Tori from ATFs oder so}
\author{JoJoJo}

\maketitle

\section{Introduction}

\begin{definition}
  Let $k ∈ ℕ$ such that $0<k≤d$ and $a ∈ (0,∞)$. Through nodal slides we can arrange the ATF on $B_{dpq}$ such that the line $x₂=a$ intersects the branch cut line between the $(k-1)$-th and $k$-th degenerated fibre. $T_k(a)$ is defined to be the fibre over the intersection point of these two lines.
\end{definition}

\begin{theorem}
  Let $U ⊂ H¹(T_k(a),ℝ) ∖ \{\text{branch cut line}\}$.
  The restriction of the displacement energy germ to $U$ is given by
  \[ \eval{S_{T_k(a)}^e}_U (x,y) = a+\max\{x,x(1-kpq)-kp²y\}\todo{so oder so ähnlich…} \]
\end{theorem}


Let \(d,p,q ∈ ℕ\) such that \(d≥\) and \(p,q\) coprime with \(1≤q<p\) or \(q=0,p=1\), and \(0<a_1<…<a_d\) real integers.
Let \(P\) be the polynomial \(P(z) = \prod_{i=1}^d (z^p-a_i)\).
Define the manifold \(M_P\) by
\[M_P = \qty{(z_1,z_2,z_3) ∈ ℂ³ \mid z₁z₂ + P(z₃)=0 } \; .\]
We define the Hamiltonian system
\[\symbf{H}(z_1,z_2,z_3) = \qty(\abs{z_3}^2, \frac{1}{2}\qty(\abs{z_1}^2-\abs{z_2}^2))\]

Let \(μ_p\) be the group of \(p\)-Th roots of unity acting on \(M_P\) by
\[μ \cdot (z₁,z₂,z₃) = \qty(μz₁,μ^{-1}z₂,μ^q z₃), \quad μ ∈ μ_p \; .\]
This is a free  action, so we can define the quotient \(B_{dpq} = M_P/μ_p\). The Hamiltonian system \(\symbf{H}\) is invariant under the action, so it descends to a Hamiltonian system on \(B_{dpq}\).

\section{Upper bound on displacement energy: Probes}

\section{Short interlude: Homology of \texorpdfstring{$B_{dpq}$}{Bdpq}}
\label{sec:homology}

In order to calculate the lower bound for the displacement energy of a torus \(L(x,y)\)\todo{definieren}, we will need to calculate a basis for \(H_2(B_{dpq},L(x,y)\).

\(B_{dpq}\) deformation retracts to the preimage of the branch cut line segment shown in \cref{fig:branch_cut_retraction}.
This can be understood as follows: If there were no critical points on the line, this would be a solid torus \(T = S¹×D²\).
We pick \((1,0),(0,1) ∈ H₁(∂T)\) to be the classes generated by \(S¹×\text{pt},\text{pt}×∂D²\) respectively.
At each critical point we collapse a loop along homology class \((p,-q)\).
Up to homotopy this is the same as attaching a disk along \((p,-q)\).
Again up to homotopy we can also require that the \(d\) discs \(D_1,…,D_d\) are attached along \(∂T\).
Let us call this space $S$.

\begin{figure}
  \centering
  \missingfigure{retaction to branch cut}
  \caption{asdf}
  \label{fig:branch_cut_retraction}
\end{figure}

Let us look at the long exact sequence of homology for the pair \((B_{dpq},L(x,y))\). This pair is homotopy equivalent to \((S,∂T)\).

\[
\begin{tikzcd}
  H_2(∂T) \ar[r,"0"] &
  H_2(S) \ar[r,hook]\ar[d,"≅"] &
  H_2(S,∂T) \ar[r]\ar[d,"≅"] &
  H_1(∂T) \ar[r]\ar[d,"≅"] &
  H_1(S) \ar[d,"≅"]
  \\
  &
  ℤ^{d-1} &
  ℤ^{d+1} &
  ℤ² &
  ℤ_p
\end{tikzcd}
\]

The first horizontal map is zero since \(∂T\) retracts to a point in \(S\).
Homology \(H_2(S)\) can be seen as follows: By contracting the solid torus \(T\) in \(S\) to a circle, we see that \(S\) is homotopic to a circle with \(d\) discs glued to its boundary by a degree \(p\) map.
So \(H_2(S)\) is generated by spheres \(\qty{S_2,…,S_d}\), \(S_k = D_1-D_{k+1}\).
\(H_2(S,∂T)\) is generated by the discs \(D_0 = \text{pt}×D²,D₁,…,D_d\). In \(B_{dpq}\), these discs can be seen, where the disc intersecting the toric boundary collapses the \((0,1)\) cycle in the toric fibre \(L(x,y)\) and the discs intersecting the critical points collapse the \((q,-p)\) cycle (see \cref{fig:homology_generating_discs}).

\begin{figure}
  \centering
  \missingfigure{Generating discs}
  \caption{asdf}
  \label{fig:homology_generating_discs}
\end{figure}

\section{Lower Bound on Displacement Energy: Minimal J-holomorphic Curves}

Let \(L(x,y)\) a fibre torus, where \((x,y)\) is not over the branch cut line. In \cite{chekanov1998} it is proved that lalalala \todo{}

Pick a tame almost complex structure $J$ on $B_{pdq}$.\todo{Or some standard structure? Does it matter?}
Let \(u\) be a non-constant J\-/holomorphic curve having possibly a boundary on \(L(x,y)\).
Then the homology class of \(u\) can be written in terms of the generators of \(H₂(B_{dpq},L(x,y))\) described in \cref{sec:homology}:
\[\qty[u] = c_0 D_0 + c_1 D_1 + ∑_{k=2}^d c_k S_k \;.\]

The symplectic area of \(u\) is then given by
\[∫_u ω = c_0 ∫_{D_0} ω + c_1 ∫_{D_1} ω \;,\]
as the symplectic area of the spheres \(S_k\) is zero.

\subsection{\texorpdfstring{\(D₀\)}{D₀} is a Minimal J-Disk in \texorpdfstring{\(B_{dpq}\)}{Bdpq}}

By a nodal slide we can move the critical points in the moment image such that they don't occur for \(\symbf{H}_2 > 2y\). By classification of toric manifolds, \(\{p ∈ B_{dpq} \mid \symbf{H_2}(p)<2y\}\) is then symplectomorphic to \(B²(2y)×ℝ×S¹\), where \(B²(a)\) is the 2-ball of area \(a\).
Here we can choose \(D_0\) to be \(\overline{B^2(y)}×\{(x,\text{pt})\}\), which is J-holomorphic with the standard almost complex structure. Our claim is that \(D_0\) is the minimal J-disc.

\subsection{\texorpdfstring{\(D_0\)}{D₀} stays a Minimal J-Disks for Suitable Embeddings}

Let \((n,a)\) be two coprime integers, \(μ_n\) the group of \(n\)-th roots of unity. Let \(μ_n\) act on \(ℂ^2\) by \(μ(z₁,z₂) = ( μ z_1,μ^a z_2)\).
Let \(A(n,a) = ℂ^2/μ_n\) be the quotient space.
This space is an orbifold, with one orbifold point at \([(0,0)]\).

We define the Hamiltonian system on \(A(n,a)\) by 
\begin{equation}
  \label{eqn:hamsysAna}
  \symbf{G}(z_1,z_2) = \frac{1}{2}\qty(\abs{z_2}², \frac{1}{n}\qty(\abs{z₁}²+a\abs{z₂}²)) \;.
\end{equation}

With this Hamiltonian system the moment polytope is a wedge with edges pointing along vectors \((1,0), (n,a)\), as seen in \cref{fig:Ana_moment_polytope}. \(A(n,a)\) has a almost complex structure \(J\) coming from the canonical complex structure on \(ℂ²\).

\begin{figure}
  \centering
  \missingfigure{\(A(n,a)\) moment polytope}
  \caption{Moment polytope of \(A(n,a)\) with given by Hamiltonian system \(G\)}
  \label{fig:Ana_moment_polytope}
\end{figure}

Let \(B(a) ∈ ℂ^n\) be the open ball in \(ℂ^n\) of radius \(\sqrt{a}\).
In \cite[Appendix~A]{chekanovschlenk2015} the following lemma is proven:
\begin{lemma}
  \label{lem:hyperannulus}
  Let \(a_+ > a_- ≥ 0\).
  Let \(u: Σ → B(a_+)∖ \overline{B(a_-)}\) be a J-holomorphic curve such that the closure of \(u(Σ)\) in \(ℂ^n\)intersects \(∂B(a_-)\).
  Then \(∫_u ω ≥ a_+ - a_-\).
\end{lemma}

We give the slight generalization:

\begin{lemma}
  \label{lem:hyperannulus2}
  Let \(a_+ > a_- > 0\), and
  \[X = \symbf{G}^{-1}\qty(\qty{c₁(0,1)+c₂(n,a) \mid a_- < c₁ + c₂ < a_+}) ⊂ A(n,a) \;,\]
  equipped with the almost complex structure of \(A(n,a)\).

  Let \(u: Σ → X\) be a J-holomorphic curve whose closure intersects
  \[\symbf{H}^{-1}\qty(\qty{c₁(0,1)+c₂(n,a) \mid a_- = c₁ + c₂}) \; .\]

  Then \(∫_u ω ≥ a_+ - a_-\).
\end{lemma}

\begin{remark}
  Suppose we have a moment polytope \(Δ\) of a (almost) toric symplectic manifold or orbifold \(\symbf{H} : M → Δ\) with two non-parallel edges given by the two primitive vectors $u₁,u₂$, as in \cref{fig:cutting_out_a_hyperannulens}.
  Suppose without loss of generality that the edges intersect in the origin.
  Then the subset
  \[X = \symbf{H}^{-1}\qty(\qty{c₁u₁+c₂u₂ \mid a_- < c₁ + c₂ < a_+})\]
  with \(a_±\) such that \(a_± u₁, a_± u₂ ∈ Δ\), can be transformed by a \(T ∈ \GL(ℤ²)\), such that \(Tu₁=(0,1), Tu₂=(n,a)\), for some coprime integers \(n,a\).

  With this transformation we can view \(X\) as a subset of \(A(n,a)\). Equipping \(M\) with an extension of the almost complex structure coming from \(A(n,a)\), we get that J-curves in \(M\) intersecting
  \[X = \symbf{H}^{-1}\qty(\qty{c₁u₁+c₂u₂ \mid a_- = c₁ + c₂})\]
  must have at least area \(a_+^2 - a_-^2\).
\end{remark}

\begin{figure}
  \centering
  \missingfigure{Hyperannulens inside a moment polytope.}
  \caption{Hyperannulens \todo{\textbf{NOOOOOOOOOOOO!}} inside a moment polytope.}
  \label{fig:cutting_out_a_hyperannulens}
\end{figure}
\begin{proof}
  Since the action of \(μ_n\) is free in \((ℂ^*)²\), the projection map \(π : (ℂ^*)² → A(n,a) ∖ \{[(0,0)]\}\) is an n-fold covering map.

  As shown below, the preimage \(π^{-1}(X)\) is \(B(na_+) ∖ \overline{B(na_-)}\), and \(u\) lifts to a J-curve \(\tilde{u}\), i.e. a curve making the diagram
  \[
    \begin{tikzcd}
      Σ' \ar[r,"\tilde{u}"] \ar[d,"\tilde{π}"] & ℂ² \ar[d,"π"] \\
      Σ \ar[r,"u"] & A(n,a)
    \end{tikzcd}
  \]
  commute, where \(\tilde{π} : Σ' → Σ\) is some n-fold covering of \(Σ\).

  Using \cref{lem:hyperannulus}, we get that the symplectic area of \(\tilde{u}\) is at least \(n(a_+ - a_-)\), and since \(\tilde{u}\) is an n-fold covering of \(u\), \(u\) has at least symplectic area \(a_+-a_-\), as desired.

  To show that \(π^{-1}(X) = B(na_+) ∖ \overline{B(na_-)}\), note that the Hamiltonian system \(\symbf{G}\) from \cref{eqn:hamsysAna}, is given by a linear transformation of the standard system on \(ℂ²\)
  \[\symbf{H}(z₁,z₂) = \frac{1}{2}\qty(\abs{z₁}²,\abs{z₂}²) \;,\]
  given by the matrix
  \[\mqty(0 & 1 \\ \frac{1}{n} & \frac{a}{n})\todo{Das ist jetzt nicht wie bei Evans linkswirkend, sondern normal…} \;,\]
  whose inverse maps \(\symbf{G}(X)\) to \(\symbf{H}(B(na_+) ∖ \overline{B(na_-)})\), and since \(π\) maps fibres to fibres, the claim follows.
\end{proof}

\iffalse
\begin{proof}
  Let \(r ∈ (a_-,a_+)\) such that the intersection \(u(Σ) ∩ ∂B_r\) is transversal.
  Then this intersection is an immersed 1-dimensional manifold, so it is a collection of immersed circles.
  Let \(γ\) be a parametrization of one of these circles.
  We choose a local holomorphic reparametrization of \(u\) as follows: 
  \begin{align*}
    \tilde{u} : S¹xI &→ ℂ^n \\
    u(t,0) &= γ(t)
  \end{align*} \todo{ja, was jetzt?}
  
  Then
  \begin{align*}
    F'(r) &= ∫_{u(Σ) ∩ ∂B_r} u^* ω \\
          &≥ ∫_{S¹} ω(\pdv{u∘s}{t},\pdv{u∘s}{r}) \dd t \\
          &= ∫_{S¹} ω(\pdv{u}{t}+\pdv{s}{t}\pdv{u}{s},\pdv{s}{r}\pdv{u}{s}) \dd t \\
          &= ∫_{S¹} \pdv{s}{r} ω(\pdv{u}{t},i \pdv{u}{t}) \dd t \\
          &= ∫_{S¹} \pdv{s}{r} \abs{\dot{γ}(t)}^2 \dd t \\
          &≥ l^2(γ) ∫_{S¹} \pdv{r}{s} \dd t
  \end{align*}

  Also
  \begin{align*}
    l^2(γ) &≥ ∫_{S¹} γ^* α_n = ∫_{S¹}⟨\dot{γ},ξ⟩ \dd t
  \end{align*}
\end{proof}
\fi



\printbibliography

\end{document}
