% vim: spelllang=en_gb

\documentclass[12pt,a4paper,draft]{scrartcl}
\usepackage{ifdraft}

% --------------------
% Set Language Options
% --------------------

\usepackage[nswissgerman,french,main=english]{babel}
\usepackage[autostyle,english=american,german=swiss]{csquotes}
\MakeOuterQuote{"}

\usepackage[shortcuts]{extdash}

% --------------
% Font & Symbols
% --------------

\usepackage{mathtools}
\usepackage[warnings-off={mathtools-colon,mathtools-overbracket}]{unicode-math}
\usepackage[oldstyle,proportional]{libertinus}

% ---------------
% Set Page Layout
% ---------------

% Get length of 65 characters
%\setlxvchars

\usepackage[driver=auto]{geometry}
% A5: 148mm × 210mm
% A4: 210mm × 297mm
\geometry{
  width=140mm,
  height=217mm,
  marginparsep=3mm,
  marginparwidth=30mm,
}
\ifdraft{\geometry{
  inner=10mm,
  marginparwidth=50mm
}}{}


% ---------------------
% Load Various Packages
% ---------------------

% Various Math Environments
\usepackage{amsthm,thmtools}
\usepackage{physics} % various shortcuts

% Bibliography
\usepackage{biblatex}
\addbibresource{bibliography.bib}

% For general figures
\usepackage[final]{graphicx}
\graphicspath{{/img}}
\usepackage{subcaption}
\usepackage{tikz}
\usetikzlibrary{babel,cd,shapes}
\tikzcdset{arrow style=math font}
\tikzset{cross/.style={cross out, draw, solid, thin, 
         minimum size=2*(#1-\pgflinewidth), 
         inner sep=0pt, outer sep=0pt},
         cross/.default={3}}

% For lists
\usepackage[shortlabels]{enumitem}

% For better Tables
\usepackage{tabularray}

% For more fine grained typesetting in final mode.
% Else set the tolerance for overfull warnings higher.
\ifdraft{\hfuzz=1.5pt}{\usepackage{microtype}}

% Links and stuff
\usepackage[final]{hyperref}
\usepackage[noabbrev,capitalize]{cleveref}

% For Todonotes
\usepackage[obeyDraft]{luatodonotes}

% --------------------------------------------
% Define Theorem Environments & Math Operators
% --------------------------------------------

\declaretheorem[numberwithin=section]{theorem}
\declaretheorem[sibling=theorem]{lemma, proposition, corollary}
\declaretheorem[sibling=theorem,style=definition]{definition, example}
\declaretheorem[sibling=theorem,style=remark]{remark}

\DeclareMathOperator{\im}{im}
\DeclareMathOperator{\Aut}{Aut}
\DeclareMathOperator{\Diff}{Diff}
\DeclareMathOperator{\GL}{GL}
\DeclareMathOperator{\HF}{HF}
\DeclareMathOperator{\HM}{HM}
\DeclareMathOperator{\Hom}{Hom}
\DeclareMathOperator{\Ext}{Ext}
\DeclareMathOperator{\Tor}{Tor}

\begin{document}
\title{Exotic Tori from ATFs oder so}
\author{JoJoJo}

\maketitle

\section{Introduction}

\begin{definition}
  Let $k ∈ ℕ$ such that $0<k≤d$ and $a ∈ (0,∞)$. Through nodal slides we can arrange the ATF on $B_{dpq}$ such that the line $x₂=a$ intersects the branch cut line between the $(k-1)$-th and $k$-th degenerated fibre. $T_k(a)$ is defined to be the fibre over the intersection point of these two lines.
\end{definition}

\begin{restatable}{theorem}{thmbdpqexotic}
    \label{thm:bdpqexotic}
  Let $U ⊂ H¹(T_k(a);ℝ) ∖ \{\text{branch cut line}\}$.
  Up to change of basis, the restriction of the displacement energy germ to $U$ is given by
  \[ \eval{\symcal{E}_{T_k(a)}}_U (x,y) = a+\min\{x,x(1-kpq)+ykp²\}\todo{so oder so ähnlich…} \]
\end{restatable}


Let $d,p,q ∈ ℕ$ such that $d≥1$ and $p,q$ coprime with $1≤q<p$ or $q=0,p=1$, and $0<a_1<…<a_d ∈ ℝ$.
Let $P$ be the polynomial $P(z) = \prod_{i=1}^d (z^p-a_i)$.
Define the manifold $M_P$ by
\begin{equation}
  \label{eqn:MP}
M_P = \qty{(z_1,z_2,z_3) ∈ ℂ³ \mid z₁z₂ + P(z₃)=0 } \; .
\end{equation}
We define the Hamiltonian system
\begin{equation}
  \label{eqn:non-toric-system}
  \symbf{H}(z_1,z_2,z_3) = \qty(\abs{z_3}^2, \frac{1}{2}\qty(\abs{z_1}^2-\abs{z_2}^2))
\end{equation}

Let $ρ_p$ be the group of $p$-th roots of unity acting on $M_P$ by
\[ρ \cdot (z₁,z₂,z₃) = \qty(ρz₁,ρ^{-1}z₂,ρ^q z₃), \quad ρ ∈ ρ_p \; .\]
This is a free  action, so we can define the quotient $B_{dpq} = M_P/ρ_p$. The Hamiltonian system $\symbf{H}$ is invariant under the action, so it descends to a Hamiltonian system on $B_{dpq}$, giving us the almost toric fibration $\symbf{H} \colon B_{dpq} → B$, where $B = \im(H)$.
As in \cite[Chapter 6]{evans2021atfs}, we can remove a ray going through the critical values in the moment image, and use the flux map to obtain a moment map $μ \colon B_{dpq} → Δ_{B_{dpq}}$ generating a Hamiltonian torus action everywhere except on the critical points, and having moment image $Δ_{B_{dpq}}$ as in \cref{fig:Bdpq_moment_image}.\todo{Kann man besser formulieren.}

The position of the nodes can be varied along this ray by nodal slides.\todo{…which only affect some ε-neighboorhood.}

\begin{figure}
  \centering
  \begin{tikzpicture}[scale=.7]
    \fill[black!5] (0,4) rectangle (8,-2);

    \draw[thick,dotted,->] (2,1) node[cross] {} -- (4,2) node[cross] {} -- (8,4) node[anchor=west] {$(p,q)$};
    \draw[thick] (0,4) -- (0,-2);
  \end{tikzpicture}
  \caption{Moment image of $B_{dpq}$ under $μ$}
  \label{fig:Bdpq_moment_image}
\end{figure}


\section{Homology of \texorpdfstring{$B_{dpq}$}{Bdpq}}
\label{sec:homology}

In order to calculate the lower bound for the displacement energy of a Lagrangian fibre torus $T(x,y) = μ^{-1}(\qty{(x,y)})$, we will need to calculate a basis for the homology $H_2\qty(B_{dpq},T(x,y))$.

$B_{dpq}$ deformation retracts to the preimage of the branch cut line segment $l$ shown in red in \cref{fig:branch_cut_retraction}, by first vertically shrinking the space onto the ray in direction $(p,q)$, and then compressing the part of the ray that is to the right of all the critical points.
\todo{Das ist etwas dumm formuliert, lohnt es sich das besser zu formulieren?}

The preimage $μ^{-1}(l)$ can be understood as follows: If there were no critical points on the line, this would be a solid torus $T = S¹×D²$.
We pick $(1,0),(0,1) ∈ H₁(∂T)$ to be the classes generated by $S¹×\text{pt},\text{pt}×∂D²$ respectively.
For each critical fibre $k ∈ \qty{1,…,d}$ we collapse a loop along the homology class $(-q,p)$, as in \cref{fig:collapse_cycles}.
Up to homotopy this is the same as attaching a disk $D_k$ along $(-q,p)$.
Again up to homotopy we can also require that the $d$ discs $D_1,…,D_d$ are attached along $∂T$.
Let us call this space $S$.

\begin{figure}
  \centering
  \begin{subfigure}{0.45\textwidth}
    \centering
      \begin{tikzpicture}[scale=.7]
      \begin{scope}[fill=black!5]
        \fill[clip] (0,4) rectangle (8,-2);

        \draw[black!30, very thick] (0,0) -- (8,4);
        \draw[red,line width=.125cm,draw opacity=.5, line cap=round] (0,0) -- (4,2) node[anchor=north] {$l$};
        \draw[thick,dotted] (2,1) node[cross] {} -- (4,2) node[cross] {} -- (8,4);
      \end{scope}
      \draw[thick] (0,4) -- (0,-2);
    \end{tikzpicture}
    \caption{Retraction to branch cut line}
    \label{fig:branch_cut_retraction}
  \end{subfigure}%
  \begin{subfigure}{0.55\textwidth}
    \includegraphics[width=\textwidth]{img/homology_collapse.png}
    \caption{The cycles marked in black (here $p=2, q=1$) are collapsed to a point.}
    \label{fig:collapse_cycles}
  \end{subfigure}
  \caption{Calculation of the homology of $B_{dpq}$}
\end{figure}

Let us look at the long exact sequence of homology for the pair $\qty(B_{dpq},T(x,y))$. This pair is homotopy equivalent to $(S,∂T)$.

\[
\begin{tikzcd}
  H_2(∂T) \ar[r,"0"] &
  H_2(S) \ar[r,hook]\ar[d,"≅"] &
  H_2(S,∂T) \ar[r]\ar[d,"≅"] &
  H_1(∂T) \ar[r]\ar[d,"≅"] &
  H_1(S) \ar[d,"≅"]
  \\
  &
  ℤ^{d-1} &
  ℤ^{d+1} &
  ℤ² &
  ℤ_p
\end{tikzcd}
\]

The first horizontal map is zero since $∂T$ retracts to a circle in $S$.
The homology $H_2(S)$ can be seen as follows: By contracting the solid torus $T$ in $S$ to a circle, we see that $S$ is homotopic to a circle with $d$ discs glued to its boundary by a degree $p$ map.
So $H_2(S)$ is generated by spheres $\qty{S_2,…,S_d}$, $S_k = D_{k-1}-D_{k}$.
$H_2(S,∂T)$ is generated by the discs $D_0 = \text{pt}×D²,D₁,…,D_d$. In $B_{dpq}$, these discs can be seen, where the disc intersecting the toric boundary collapses the $(0,1)$ cycle in the toric fibre $T(x,y)$ and the discs intersecting the critical points collapse the $(-q,p)$ cycle (see \cref{fig:homology_generating_discs}).
The elements $S_2,\ldots,S_d \in H_2(B_{dpq})$ can be realized by embedded Lagragian spheres fibering over the segments between the nodes in the ATF -- these are so-called \emph{visible Lagrangians}, see \cite[section 7.4]{evans2021atfs}.
The boundary map $∂ \colon H_2(S,∂T) → H_1(∂T)$ is given by $\partial D_0 = (0,1),\, \partial D_i = (-q,p)$, meaning that the last horizontal map $H_1(∂T) → H_1(S)$ maps $(0,1)$ to the generator of $ℤ_p$.


\begin{figure}
  \centering
  \begin{tikzpicture}
    \fill[black!5] (0,4) rectangle (7,-1);

    \coordinate (xy) at (2.7,3.5);
    \node[anchor=south] at (xy) {$T(x,y)$};
    \fill (xy) circle[radius=.05];
    \draw (xy)
           .. controls +(0,0) and +(-.5,1) .. node[anchor=east, near end] {$D₁$} (2,1)
      (xy) .. controls +(0,0) and +(-.5,1) .. node[anchor=west, near end] {$D₂$} (4,2)
      (xy) .. controls +(0,0) and +(-.5,1) .. node[anchor=west, near end] {$D₃$} (6,3)
      (xy) .. controls +(0,0) and +(1,0) .. node[anchor=south, near end] {$D₀$} (0,2.5);

    \draw[ultra thick, blue, draw opacity=.3] (2,1) -- node[anchor=north] {$S₂$} (4,2);
    \draw[ultra thick, purple, draw opacity=.3] (4,2) -- node[anchor=north] {$S₃$} (6,3);

    \draw[thick,dotted] (2,1) node[cross] {} -- (4,2) node[cross] {} -- (6,3) node[cross] {} -- (7,3.5);

    \draw[thick] (0,4) -- (0,-1);
  \end{tikzpicture}

  \caption{The disks $D₀, …, D_d$ generating the homology $H₂\qty(B_{dpq}, T(x,y))$}
  \label{fig:homology_generating_discs}
\end{figure}

\section{Lower Bound on Displacement Energy: Minimal J-holomorphic Curves}
\label{sec:lower_bound}

We want to use the following Theorem by Chekanov \cite{chekanov1998} for a lower bound of the displacement energy

\begin{theorem}
  \label{thm:chekanov}
  Let $(X,ω,J)$ be a tame symplectic manifold with $ω$-tame almost complex structure $J$. Let $L ⊂ X$ be a compact Lagrangian submanifold. Then the displacement energy satisfies
  \[e(L) ≥ \min\qty{σ_D(X,L,J),σ_S(X,J)}\]
\end{theorem}

Fibres $T(λ p,λ q)$ for $λ > 0$ are monotone whenever they are not on a node. We compute displacement energy of the non-monotone fibres.\todo{das müsste man auch noch zeigen oder verrefferenzen, ist aber irgendwie irrelevant…}
Note that $T(x_0,y_0)$ yields a well-defined torus up to Hamiltonian isotopy, i.e.\ it is independent of the nodal slides, see \todo{referenz, wahrscheinlich nicht der richtige ort für diesen kommentar.}.

To use \cref{thm:chekanov}, we can choose a suitable almost complex structure on $B_{dpq}$.

We do so in the following way: During the construction of $B_{dpq}$, we defined $B_{dpq}$ as the quotient of $M_P$ under the free action of $ρ_p$. As a algebraic hypersurface $M_P$ naturally inherits the complex structure of $ℂ³$, and we use $J$ to denote it on $M_P$ as well as on its quotient $B_{dpq}$.

The first coordinate of $\symbf{H}$ in \cref{eqn:non-toric-system}, is a plurisubharmonic function on $ℂ³$, as $-\dd \dd^ℂ \symbf{H}₁ = 2 \dd{x_3} ∧ \dd{y_3} ≥ 0$. As $M_P$ is an algebraic hypersurface, this restricts to a plurisubharmonic function on $M_P$, and also descends to the quotient $B_{dpq}$ as a plurisubharmonic function.

In particular, any J-holomorphic curve $u\colon Σ → B_{dpq}$ satisfies the maximum principle with respect to $\symbf{H}₁$, i.e.\ if $\symbf{H}₁ ∘ u$ has a maximum, it must be on $∂Σ$, and if $\symbf{H}₁ ∘ u$ is non-zero then it must be constant.\footnote{The latter is obtained by applying the maximum principle to $\frac{1}{\symbf{H}₁ ∘ u}$.}

Let $b = (b₁,b₂) ∈ \im(\symbf{H})$ and set $T = \symbf{H}^{-1}(b)$. Let $u: (Σ,∂Σ) → (B_{dpq},L)$ be a J-holomorphic curve. We have $\symbf{H}₁ ∘ u ≤ b₁$ by the maximum principle. This means that if we have $b₁ ≤ \symbf{H}₁(aᵢ)$ for all $1≤i≤d$, then $u$ is entirely contained in the region $\symbf{H}^{-1}(\qty{\symbf{H}₁ ≤ b₁})$ which is homotopic to a solid torus rel $T$.

Since the homology of this space is isomorphic to $ℤ$ and generated by $D₀$ as in \cref{sec:homology}, and the area of a J-disc rel $L$ only depends on its homology class, we have that
\[σ_D(X,T,J) ≥ ∫_{D₀} ω  = μ₁(T)\]

Since $B_{dpq}$ contains no J-holomorphic spheres,\footnote{Its homology is generated by Lagrangian spheres, thus any closed J-holomorphic curve has symplectic area $0$ and must be constant.} we have proven the following:

\begin{lemma}
  \label{lem:lower_bound}
  If $b₁ < \symbf{H}₁(aᵢ)$ for all $1 ≤ i ≤ d$, then
  \[e(B_{dpq},L) ≥ σ_D(X,L,J) ≥ ∫_{D₀} ω  = μ₁(L) \; .\]
\end{lemma}

\section{Upper bound on displacement energy: Probes}

We want to use the method of probes to calculate an upper bound on the displacement energy of certain toric fibres $T(x,y)$, i.e. we give a Hamiltonian diffeomorphism that displaces the $T(x,y)$ from itself. The method to construct such a Hamiltonian diffeomorphism is the method of probes, that was first introduced by McDuff \cite{mcduff2011displacing} for exactly the same purpose. Probes were originally introduced in the context of toric symplectic manifolds.\\
We quickly recall the main ideas of the method and then explain how they are applied in our situation. Assume that $(X^{2n},\omega)$ is a toric symplectic manifold with moment map $\mu:X\to ℝ^n$ and moment polytope $\Delta$. A \textit{probe} $p_\lambda(w)$ is a half open line segment contained in $\Delta$, starting at a point $w \in \Delta$, where $w$ is lying on a facet of $\Delta$, pointing in the direction of $\lambda \in ℤ^n$, where the direction  $\lambda$ is integrally transverse to the facet and such that the line segment intersects the boundary of the polytope $\Delta$ only in the point $w$. In \cite[Lemma 2.4.]{mcduff2011displacing} it is shown that if a toric fibre $\mu^{-1}(u)$ lies over a probe $p_\lambda(w)$, where $u\in \text{int}\Delta$ is strictly less than halfway along the probe, then it is displaceable. It is worthwhile to note that this Hamiltonian displacement is local in nature. To see this one realises that the definition probes implies that one can find an appropriate $Gl_n(\mathbb{Z})$ transformation of the $\mathbb{R^n}$ such that the facet $F$, from which the probe emanates, lies on the hyperplane $\{x_1=0\}$ and such that the direction of the probe is given by $\lambda=(1,0,\ldots,0)$. Furthermore, there is a corresponding Darboux chart on $X$ with coordinates $z_1,\ldots,z_n$ such that the preimage under the moment map of the probe is given by
\begin{equation*}
    \mu^{-1}(p_\lambda(w))=\left\{z\in \mathbb{C}^n \mid \abs{z_1}^2<2a, \abs{z_i}=\text{const for }i\neq 1\right\},
\end{equation*}
where $a \in \mathbb{R}$ is the affine length of the probe. By abuse of notation this information is condensed in the following diagram of symplectic reduction
\[
\begin{tikzcd}
  \left(\mu^{-1}(p_\lambda(w)),\left.\omega\right|_A\right)\ar[r,hook]\ar[d]&
  \left(\mu^{-1}(U),\left.\omega\right|_U\right)
  \\
  \left(\mathbb{D}^2(a),\omega_\text{std}\right),
\end{tikzcd}
\]
where $\mathbb{D}^2(a)\subseteq \mathbb{C}$ is the open disk enclosing area $2\pi a$. Now a circle $S^1(b)\vcentcolon=\{z\in \mathbb{C}\mid \abs{z}^2=2b\} \subseteq \mathbb{D}^2(a)$ is displaceable by a Hamiltonian diffeomorphism of $\mathbb{D}^2(a)$ if and only if $b<\frac{a}{2}$. The Hamiltonian generating this diffeomorphism can be extended to all of $\mathbb{C}$ in such a way that it can be lifted to $\mu^{-1}(U)$ in a compactly supported manner.\\
It is clear that this discussion construction extends to the case of almost toric fibrations and does trivially so, if the probe does not intersect the branch cut line.\\
Now assume that everything is arranged as in the assumptions and notations of \cref{thm:bdpqexotic}. In particular the moment image associated to this setup is shown in \cref{fig:Bdpq_moment_image}. Using the method of probes we show the upper bound on the displacement energy, i.e. 

\begin{restatable}{lemma}{lemmabdpqexoticupper}
    \label{thm:lemmabdpqexoticupper}
  Let $U ⊂ H¹(T_k(a);ℝ) ∖ \{\text{branch cut line}\}$.
  Up to change of basis, the restriction of the displacement energy germ to $U$ is bounded from above by
  \[ \eval{\symcal{E}_{T_k(a)}}_U (x,y) \leq a+\min\{x,x(1-kpq)+ykp²\} \]
\end{restatable}

\begin{proof}
    Recall that $T_k(a)$ is the fibre over $(a,b) \in Δ_{B_{dpq}}$, where $b\vcentcolon=\frac{q}{p}a$, such that there are $k$ nodes below on the branch cut line. After mutating $k$ times the base point of $T_k(a)$ is contained in a neighbourhood that admits a regular torus fibration. The mutation process yields a new moment map $\overline{μ}$ and moment polytope $\overline{Δ_{B_{dpq}}}$. \todo{Draw the correct picture and show the neighbourhood that is meant.} Now a neighbourhood of $0 ∈ H¹(T_k(a);ℝ)$ can be identified with a neighbourhood $V ⊂ \overline{Δ_{B_{dpq}}}$ of $(a,\frac{qa}{p}) = μ(T_k(a))$, such that $V \subset U$. Choose generators of $H¹(T_k(a);ℝ)$ corresponding to the components of the moment map $\overline{\mu}=(\overline{\mu_1},\overline{\mu_2})$, i.e. generators that are orthogonal to those. Using this identification a point $(x,y) ∈ H¹(T_k(a);ℝ)$ close to $0$ is sent by the versal deformation to $(a + x, b + y)$ in $\overline{Δ_{B_{dpq}}}$. Now the goal is to estimate the displacement energy of fibres over these points. This is achieved by the method of probes. In the moment image $\overline{Δ_{B_{dpq}}}$ horizontal probes can be used to displace all the regular toric fibres. Denote the Lagrangian torus fibres over $(a + x, b + y)$ by $T_{x,y}$. This yields 
    \[ e(B_{dpq},T_{x,y})\leq a+x. \]
    If it is above the branch cut ray,  the mutation leaves it unaffected, and the displacement energy is $x+a$. Note that this calculation involved a choice of basis of the cohomology vector spaces, but that the displacement energy germ is intrinsically given.\footnote{For details on displacement energy germs and versal deformations compare \cite{chekanovschlenk2015} and \cite{brendel2020real}} To compare the estimates we get for different choices of $a$'s we therefore apply the inverse mutation to get back to the original moment map $\mu$. The points above the branch cut are unaffected by the mutation but to the points below the branch cut line we have to apply the inverse of the anticlockwise monodromy matrix.\todo{Maybe include the calculations here?} A point $(a + x, b + y)$ below the branch cut line is send to $(a,b) + (x(1-kpq)+kp²y, -kq²x + y(kpq +1))$. Hence the displacement energy germ calculated in these coordinates is estimated by 
    \[ \eval{\symcal{E}_{T_k(a)}}_U (x,y) \leq a+\min\{x,x(1-kpq)+ykp²\}. \]
\end{proof}


\section{Proof of non-embedded version}

\todo[inline]{ !!! Hier wird noch viel Händewedeln betrieben.}

\begin{lemma}
  \label{lem:bdq_displacement}
  Let $(x,y) ∈ Δ_{B_{dpq}}$ not on the branch cut ray. Then the displacement energy of $T = μ^{-1}(x,y)$ is equal to $x$.
\end{lemma}

\begin{proof}
  In the proof we will be wanting to move the nodes around freely by nodal slides without affecting the displacement energy. The following makes this possible:
  By \cite[Theorem 8.9]{evans2021atfs}, we can pick a contractible neighbourhood of the sliding nodes, such that the branch cut induces a fibred symplectomorphism outside that neighbourhood. Choosing a neighbourhood not intersecting $T$, we may freely apply nodal slides without affecting the displacement energy of $T$.

  For the lower bound, we first use a nodal slide to make room for a probe (see\todo{Picture}). Using that, we get the upper bound $e(T)≤x$.
  
  Using a nodal slide together with \cite[Corollary 5.4]{symington2002FourDF} which determines the symplectic manifold over a almost toric base up to symplectomorphism, we can modify the polynomial $P(z) = Π_{i=1}^d (z-a_i)$, while keeping $\symbf{H}(T) = {b}$.
  Thus we can choose the nodal slide so that we have that $b_1 = \symbf{H}_1(T) ≤ \symbf{H}_1(a_i)$ for all $ 1 ≤ i ≤ d$.
  So we can apply \cref{lem:lower_bound}, and to get the lower bound $e(T) ≥ x$.
\end{proof}

We are now ready to prove

\thmbdpqexotic*

Recall that $T_k(a)$ is the fibre over $(a,\frac{q}{p}a)$ such that there are $k$ nodes below on the branch cut ray.

\begin{proof}[Proof of \cref{thm:bdpqexotic}]
  Using mutations at the $k$ nodes to the left of $T_k(a)$, we get a new moment map $\overline{μ}$ and moment polytope $\overline{Δ_{B_{dpq}}}$ such that $T_k(a)$ is a regular \todo{this is the wrong word...} fibre of $\overline{μ}$ (Instead of sitting on a branch cut line).
  Then using a versal deformation \todo{whooo magic word}, we can identify a neighbourhood $U$ of $0 ∈ H¹(T_k(a);ℝ)$ with a neighbourhood $V ⊂ \overline{Δ_{B_{dpq}}}^{\text{reg}}$ of the basepoint $(a,\frac{qa}{p}) = μ(T_k(a))$, where we identify points in $\overline{Δ_{B_{dqp}}}^{reg}$ with their preimages, which are Lagrangian tori.

  Choose generators in $H¹(T_k(a);ℝ)$ corresponding to $μ_1,μ_2$\todo{wie genau?}. Then $(x,y) ∈ H¹(T_k(a);ℝ)$ is sent by versal deformation to $(a + x, \frac{q}{p}a + y)$ in $\overline{Δ_{B_{dpq}}}$.
  If it is above the branch cut ray, the mutation leaves it unaffected, and the displacement energy is $x+a$.
  If it is below the branch cut ray, the mutation sends it to $(a,\frac{q}{p}a) + (kp²y -x(kpq -1), -kq²x + y(kpq +1))$, and the displacement energy is $a +kp²y - x(kpq-1) = a + x + kp(py-qx)$ where the last term is negative, so we get the desired result.
\end{proof}


\section{Small Buffer Zone Lemma}

Let $(M,ω,J)$ be a symplectic manifold with $ω$-tame almost complex structure $J$, and denote by $g$ the associated Riemannian metric, and $u\colon Σ → M$ be a J-holomorphic curve.

Then monotonicity (e.g. \cite[Proposition 4.3.1 (ii)]{sikorav1994}) gives us the following statement:

\begin{lemma}[Monotonicity]
  \label{lem:monotonicity}
  Let $x ∈ \im(u)$ and $B = B_r(x)$ the open ball w.r.t.\ $g$. If $u(∂Σ) ⊂ M ∖ B$ then we have
  \[∫_u ω ≥ \frac{r²}{4π}\]
\end{lemma}

As a simple corollary we get:

\begin{corollary}
  \label{cor:small_buffer}
  Suppose $∂M = ∂M^+ ⊔ ∂M^-$, with $∂M^±$ non-empty, $∂Σ = ∂Σ^+ ⊔ ∂Σ^-$, with $∂Σ^±$ non-empty and $u\colon (Σ,∂Σ) → (M,∂M)$ J-holomorphic with $u(∂Σ^±) ⊂ ∂M^±$, then
  \[∫_u ω ≥ \frac{d(∂M^+,∂M^-)^2}{16π} \; .\]
\end{corollary}

\begin{proof}
  Pick a path $γ$ from $∂Σ^-$ to $∂Σ^+$. By the intermediate value theorem there is $t$ s.t.\ $d((u ∘ γ) (t),∂M^+) = d((u ∘ γ)(t), ∂M^-)$. Picking $x = (u ∘ γ)(t)$ and using the triangle inequality we have that $B = B_{d(∂M^+,∂M^-)/2}(x) ⊂ M ∖ ∂M $. Applying \cref{lem:monotonicity} to $B$ we get the desired result.
\end{proof}

\section{Proof of small embedded version}

\todo[inline]{AHHHHHHHHHHHHHHH}

\begin{theorem}
  Let $(M,ω)$ be a compact tame symplectic 4-manifold. Let $B_{dpq}(a) ⊂ B_{dpq}$ be the preimage of a convex open neighbourhood $U$ of $(0,0)$ and the nodes in $Δ_{B_{dpq}}$ such that $\sup\{\tilde{a} | T_k(\tilde{a}) ⊂ B_{dpq}(a)\} = a$, and $φ \colon B_{dpq}(a) → M$ an embedding. Then there exists $ε>0$ such that for all $\tilde{a}<ε$, $φ(T_k(\tilde{a})) ⊂ M$ has the same displacement energy germ as $T_k(\tilde{a}) ⊂ B_{dpq}$.
\end{theorem}

\begin{proof}
  Note that to show the result, we only need to show \cref{lem:bdq_displacement} for "the embedded case".

  We want to proceed in a similar way as \cref{lem:bdq_displacement}. To be able to do that, we need to be able to displace Lagrangian tori near $T_k(\tilde{a})$ by probes. Since $μ(B_{dpq}(a))$ is convex, we can choose an $ε_1>0$ such that this is possible for all.

  Next we want to establish the lower bound using \cref{thm:chekanov} and \cref{cor:small_buffer}. Choose a small two-sided collar $ι \colon μ^{-1}(∂U) × (-1,1)$, and $B = ι(μ^{-1}(∂U), (-1,0))$ an inner collar of $μ^{-1}(U)$.
  Then, since $μ^{-1}(∂U)$ is compact, $B$ has two disjoint boundary components $∂B^±$ and $d(∂B^+,∂B^-) > 0$ for any metric $d$.

  Choose the almost complex structure $J_0$ on $B_{dpq}(a)$ coming from $ℂ^3$ as in \cref{sec:lower_bound}. Then using \cref{cor:small_buffer} on $B$, we get that any $J_0$-holomorphic disc crossing $B$ must have area at least $\frac{d(∂B⁺,∂B⁻)²}{16 π} = ε_2$.

  Take an extension $J$ of $J_0$ to all of $M$, and let $ε_3 = σ_S(M,J)$.

  Now for all fibres $T$ with $μ_1(T) ≤ \min\{ε_1,ε_2,ε_3\}$ and $\symbf{H}_1 ≤ a_1$,\footnote{The latter can always be achieved by a nodal slide} \cref{lem:lower_bound} works, and the upper bound is given by the probe.
\end{proof}



\section{Big Buffer Zone Lemma}

Let $(n,a)$ be two coprime integers, $ρ_n$ the group of $n$-th roots of unity. Let $ρ_n$ act on $ℂ^2$ by $ρ(z₁,z₂) = ( ρ z_1,ρ^a z_2)$.
Let $A(n,a) = ℂ^2/ρ_n$ be the quotient space.
This space is an orbifold, with one orbifold point at $[(0,0)]$.
The space $S³/ρₙ$ is the lens space $L(n,a)$, so $A(n,a)$ is the cone over $L(n,a)$.

We define the Hamiltonian system on $A(n,a)$ by 
\begin{equation}
  \label{eqn:hamsysAna}
  \symbf{G}(z_1,z_2) = \frac{1}{2}\qty(\abs{z_2}², \frac{1}{n}\qty(\abs{z₁}²+a\abs{z₂}²)) \;.
\end{equation}

With this Hamiltonian system the moment polytope is a wedge with edges pointing along vectors $(1,0), (n,a)$, as seen in \cref{fig:Ana_moment_image}.
$A(n,a)$ has a almost complex structure $J$ descending from the canonical complex structure on $ℂ²$.

We have the commutative diagram
\begin{equation}
  \label{eqn:Ana_commdiag}
  \begin{tikzcd}
    ℂ² \ar[r,"π"] \ar[d,"\symbf{H}"] & A(n,a) \ar[d, "\symbf{G}"] \\
    Δ_{ℂ²} \ar[r,"L_{A(n,a)}"] & Δ_{A(n,a)}
  \end{tikzcd}
\end{equation}
where the left vertical map $\symbf{H}$ is given by $(z₁,z₂) ↦ \frac{1}{2}\qty(\abs{z_1}²,\abs{z_2}²)$, and the bottom map is a linear transformation given by the matrix
\[ L_{A(n,a)} = \mqty(0 & 1\\\frac{1}{n} & \frac{a}{n})\;.\]

\begin{figure}
  \centering
  \begin{tikzpicture}[scale=1.5]
    \draw[thick,fill=black!5,->] (0,2) -- (0,0) -- (3,2) node[anchor=south] {$(n,a)$};
  \end{tikzpicture}
  \caption{Moment image of $A(n,a)$ with given by Hamiltonian system $G$}
  \label{fig:Ana_moment_image}
\end{figure}

Let $B(a) ∈ ℂ^n$ be the open ball in $ℂ^n$ of radius $\sqrt{\frac{a}{π}}$.
In \cite[appendix~A]{chekanovschlenk2015} the following lemma is proven:
\begin{lemma}
  \label{lem:hyperannulus}
  Let $a_+ > a_- ≥ 0$.
  Let $u \colon Σ → B(a_+)∖ \overline{B(a_-)}$ be a J-holomorphic curve such that the closure of $u(Σ)$ in $ℂ^n$intersects $∂B(a_-)$.
  Then $∫_u ω ≥ a_+ - a_-$.
\end{lemma}

We give the slight generalization:

\begin{lemma}
  \label{lem:hyperannulus2}
  Let $a_+ > a_- > 0$, and
  \[X = \symbf{G}^{-1}\qty(\qty{c₁(0,1)+c₂(n,a) \mid a_- < c₁ + c₂ < a_+}) ⊂ A(n,a) \;,\]
  equipped with the almost complex structure of $A(n,a)$.

  Let $u \colon Σ → X$ be a J-holomorphic curve whose closure intersects
  \[\symbf{G}^{-1}\qty(\qty{c₁(0,1)+c₂(n,a) \mid a_- = c₁ + c₂}) \; .\]

  Then $∫_u ω ≥ a_+ - a_-$.
\end{lemma}

\begin{remark}
  \label{rem:hyperannulus3}
  Suppose we have a moment polytope $Δ$ of a (almost) toric symplectic manifold or orbifold $\symbf{H} \colon M → Δ$ with two non-parallel edges given by the two primitive vectors $u₁,u₂$, as in \cref{fig:cutting_out_a_hyperannulens}.
  Suppose without loss of generality that the edges intersect in the origin.
  Then the subset
  \[X = \symbf{H}^{-1}\qty(\qty{c₁u₁+c₂u₂ \mid a_- < c₁ + c₂ < a_+})\]
  with $a_±$ such that $a_± u₁, a_± u₂ ∈ Δ$, can be transformed by a $T ∈ \GL(ℤ²)$, such that $Tu₁=(0,1), Tu₂=(n,a)$, for some coprime integers $n,a$.

  With this transformation we can view $X$ as a subset of $A(n,a)$. Equipping $M$ with an extension of the almost complex structure coming from $A(n,a)$, we get that J-curves in $M$ intersecting
  \[\symbf{H}^{-1}\qty(\qty{c₁u₁+c₂u₂ \mid a_- = c₁ + c₂})\]
  must have at least area $a_+ - a_-$.
\end{remark}

\begin{figure}
  \centering
  \missingfigure{Hyperannulens inside a moment polytope.}
  \caption{Hyperannulens \todo{\textbf{NOOOOOOOOOOOO!}} inside a moment polytope.}
  \label{fig:cutting_out_a_hyperannulens}
\end{figure}
\begin{proof}
  Since the action of $ρ_n$ is free in $(ℂ^*)²$, the projection map $π \colon (ℂ^*)² → A(n,a) ∖ \{[(0,0)]\}$ is an n-fold covering map.

  The using the commutative diagram \ref{eqn:Ana_commdiag}, we can compute the preimage
  \[
    π^{-1}(X) = (\symbf{H} ∘ L_{A(n,a)})^{-1}(\symbf{G}(X)) = B(na_+) ∖ B(na_-)\;.
  \]

  A J-curve $u\colon Σ → A(n,a)$ lifts to a J-curve $\tilde{u}$, i.e. a curve making the diagram
  \[
    \begin{tikzcd}
      Σ' \ar[r,"\tilde{u}"] \ar[d,"\tilde{π}"] & ℂ² \ar[d,"π"] \\
      Σ \ar[r,"u"] & A(n,a)
    \end{tikzcd}
  \]
  commute, where $\tilde{π} \colon Σ' → Σ$ is some n-fold covering of $Σ$.

  Using \cref{lem:hyperannulus}, we get that the symplectic area of $\tilde{u}$ is at least $n(a_+ - a_-)$, and since $\tilde{u}$ is an n-fold covering of $u$, $u$ has at least symplectic area $a_+-a_-$, as desired.

\end{proof}



\printbibliography

\end{document}
