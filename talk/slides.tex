\documentclass[aspectratio=169,handout]{beamer}

\usetheme[sectionpage=none,
progressbar=head,
block=fill]%
{metropolis}


\usepackage[nswissgerman,french,main=english]{babel}
\usepackage[autostyle,english=american,german=swiss]{csquotes}
\MakeOuterQuote{"}

% --------------
% Font & Symbols
% --------------

\usepackage{mathtools}
\usepackage[warnings-off={mathtools-colon,mathtools-overbracket}]{unicode-math}
\usepackage[oldstyle,proportional]{libertinus}
\usepackage{amssymb}

% ---------------------
% Load Various Packages
% ---------------------

% Various Math Environments
\usepackage{physics} % various shortcuts

% Bibliography
\usepackage{biblatex}

% For general figures
\usepackage{graphicx}
\usepackage{tikz}
\usetikzlibrary{babel,cd,hobby,shapes.misc,3d}
\tikzcdset{arrow style=math font}
\tikzset{cross/.style={cross out, draw, solid, thin, 
         minimum size=2*(#1-\pgflinewidth), 
         inner sep=0pt, outer sep=0pt},
         cross/.default={3}}
\tikzset{torus/.pic={
    \draw (0,0) circle [x radius=1,y radius=0.5]
      (0,0.4) +(-135:0.5) arc [start angle=-135,end angle=-45,radius=0.5]
      (0,-0.35) +(60:0.4) arc [start angle=60,end angle=120,radius=0.4];
    }
  }

% For lists
\usepackage[shortlabels]{enumitem}

% For better Tables
\usepackage{tabularray}

\usepackage[obeyDraft]{luatodonotes}

% --------------------------------------------
% Define Theorem Environments & Math Operators
% --------------------------------------------

\DeclareMathOperator{\im}{im}
\DeclareMathOperator{\Aut}{Aut}
\DeclareMathOperator{\Diff}{Diff}
\DeclareMathOperator{\GL}{GL}
\DeclareMathOperator{\HF}{HF}
\DeclareMathOperator{\HM}{HM}
\DeclareMathOperator{\Hom}{Hom}
\DeclareMathOperator{\Ext}{Ext}
\DeclareMathOperator{\Tor}{Tor}
\DeclareMathOperator{\Ham}{Ham}
\DeclareMathOperator{\Symp}{Symp}
\DeclareMathOperator{\Flux}{Flux}

\begin{document}
\title{Toric Fibres in Local Models of Almost Toric Fibrations}
\author{Joel Schmitz \emph{(speaker)}, Joé Brendel, Johannes Hauber}
\date{\today}

\maketitle

\section{Introduction}

\begin{frame}{Classification of Lagrangian Tori in $ℂ$}
  \begin{center}
    \begin{tikzcd}[ampersand replacement=\&]
      \begin{tikzpicture}[use Hobby shortcut,scale=0.3,baseline=0]
        \draw[->, very thin] (-4,0) -- (4,0);
        \draw[->, very thin] (0,-4) -- (0,4);

        \filldraw [fill opacity=0.2,fill=green,text opacity=0.7] ([closed]2,1) .. (-3,2) .. (-3,-2) .. (0,0.2) .. (3,-2) (-3,0) node[anchor=south west] {$a$};
      \end{tikzpicture}%
      \arrow[r] \&%
      \begin{tikzpicture}[scale=0.3,baseline=0]
        \visible<2->{
        \draw[->, very thin] (-4,0) -- (4,0);
        \draw[->, very thin] (0,-4) -- (0,4);

        \filldraw[fill opacity=0.2,fill=green,text opacity=0.7] (0,0) circle [radius=3] node[anchor=south east] {$a$};
        \draw (0,0) (-30:3) node[anchor=west]{$S^1(a)$};
        }
      \end{tikzpicture}
    \end{tikzcd}
  \end{center}
  \begin{definition}<3->
    Two Lagrangians $L,L' ⊂ (X,ω)$ are Hamiltonian isotopic ($L ≅ L'$) if there is $φ ∈ \Ham(X,ω)$ with $φ(L) = L'$.
  \end{definition}

  \begin{definition}<4->[Product Torus]
    $T(a_1,…,a_n) = S^1(a_1) × … × S^1(a_n) ⊂ ℂ^n$
  \end{definition}
\end{frame}

\begin{frame}{Classification of Lagrangian Tori in $ℂ^n$}
  Are there Lagrangian tori in $ℂ^n$ that are \emph{not} Hamiltonian isotopic to product tori?

  \begin{description}
    \item<2->[$ℂ$] No. Everything is Hamiltonian isotopic to some $S¹(a)$.
    
    \item<3->[$ℂ²$] Yes. There is 1-parameter family of monotone tori. (Chekanov-Torus)
    
    \item<4->[$C^n, n ≥ 3$] There exist infinitely many different families of equivalence classes different to product tori. (Chekanov-Schlenk 10, Auroux 15, Brendel 23)
    
%    \item<5->[$ℂP^2$] Yes, countably infinitely many monotone distinct tori. (Vianna 14)
  \end{description}
\end{frame}

\begin{frame}{Auroux System \& Chekanov Torus}
  $ℂ²$ moment map $μ \colon ℂ² → ℝ²$, $μ(z_1,z_2) = (\abs{z_1}², \abs{z_2}²)$
  \hspace{2cm}
  \begin{tikzpicture}[scale=1.5,baseline=1.2cm]
    \only<1>{\fill[opacity=0.1] (0,1) rectangle (1,0);
\draw[thick] (0,1) -- (0,0) -- (1,0);}
  \end{tikzpicture}
  \vspace{-1.0cm}

  \pause

  \textbf{Auroux system}: $\symbf{H} \colon ℂ² → ℝ²$ given by $\symbf{H}(z_1,z_2) = (\abs{z_1 z_2 - c}^2, \abs{z_1}^2 - \abs{z_2}^2)$.

  \begin{tikzpicture}
    \fill[blue,opacity=0.1] (0,1) rectangle (3,-1);
    \draw[thick] (0,1) -- (0,-1) (1,0) node[cross] {} node[anchor=west] {$(\abs{c}²,0)$};
    \uncover<4>{\fill (0.5,0) circle [radius=1pt] node[anchor=south] {$T_{Ch}$};}
  \end{tikzpicture}
  \hspace{2cm}
  \pause
  \begin{tikzpicture}[scale=2]
    \begin{scope}[canvas is xz plane at y=0]
      \draw (2,.5) node[anchor=west] {$z_1 z_2$};
      \clip (-1,-1) rectangle (2,1);
      \fill[opacity=0.1, blue] (-1,-1) rectangle (2,1);
      \foreach \x in {0.25,0.5,...,2.0}
        \draw[very thin, gray] (1,0) circle [radius=\x];
      \fill (0,0) node[cross] {} (1,0) circle [radius=.5pt] node[anchor=west] {$c$};
    \end{scope}
    \foreach \x in {0.8,1.2} {
      \begin{scope}[canvas is xz plane at y=\x]
        \draw (0,0) circle[radius=0.15] (0.5,0) circle[radius=0.15] (1.5,0) circle[radius=0.15];
      \end{scope}
    }
    \begin{scope}[canvas is xz plane at y=1.0]
      \foreach \x in {0.5,1.5}
        \draw (\x,0) circle [radius=0.07];
      \draw[thin, gray] (1,0) circle [radius=0.5-0.07] circle [radius=0.5+.07];
    \end{scope}
    \begin{scope}[canvas is xy plane at z=0]
      \draw (.15,0.8) -- (-.15,1.2) (.15,1.2) -- (-.15,0.8);
      \foreach \x in {0.5,1.5}
        \draw (\x-.15,0.8) .. controls (\x-.05,1.0) .. (\x-.15,1.2) (\x+.15,1.2) .. controls (\x+.05,1.0) .. (\x+.15,0.8);
      \foreach \x in {0,0.5,1.5}
        \draw[very thin,gray] (\x,0) -- (\x,0.7);
    \end{scope}
  \end{tikzpicture}

  \pause

  \begin{definition}[Chekanov Torus]
    $T_{Ch}(a) ≔ \symbf{H}^{-1}(\{(a,0)\})$ for $0 < a < \abs{c}^2$
  \end{definition}
\end{frame}

\begin{frame}{Flux map}
  $L ⊂ (X,ω)$ Lagrangian torus\\
  $\symcal{L} = \qty{L' \; \text{Lagrangian isotopic to} \; L}$\\
  $\hat{\symcal{L}} = \qty{Λ \; \text{Lagrangian isotopy starting at} \; L}/\qty{\text{end point preserving homotopy}}$

  \pause

  \parbox{6cm}{
  \begin{align*}
    \Flux \colon \hat{\symcal{L}} &→ H^1(L;ℝ) \\
    ⟨\Flux(Λ), ξ ⟩ &= ∫_{C_ξ} ω 
  \end{align*}
  }
  \hspace{1cm}
  \begin{tikzpicture}[scale=0.8,every pic/.style={scale=0.8},baseline=0]
    \draw (-2,-0.5) .. controls ++(2,-0.5) and ++(-2,0) .. node[anchor=south] {$Λ$} (2,0.5);
    \fill[green!50!black] (-2,-.5) circle [radius=1pt] node [anchor=east] {$L$};
    \fill[purple] (2,.5) circle [radius=1pt] node[anchor=west] {$L'$};

    \draw[green!50!black] (-2,1) pic {torus};
    \draw[red] (-2,0.7) circle [x radius=0.1,y radius=0.2] node[anchor=east] {$ξ$};

    \draw[orange!80!black] (-2,0.9) .. controls ++(2,-0.5) and ++(-2,0) .. node[anchor=south] {$C_ξ$} (2,1.9)
    (-2,0.5) .. controls ++(2,-0.5) and ++(-2,0) .. (2,1.5);
    \draw[orange!80!black] (2,1.7) circle [x radius=0.1,y radius=0.2];

    \draw[purple] (2,2) pic {torus};
  \end{tikzpicture}
\end{frame}

\begin{frame}{Flux Map \& Versal Deformations}
  \begin{theorem}[Ono]
    Lagrangian tori in $T^* L$ are classified by $\Flux \colon \symcal{L} → H^1(L,ℝ)$ up to Hamiltonian isotopy.
  \end{theorem}

  \pause

  For $L'$ $C¹$-close to $L ⊂ X$, ∃ a Weinstein-neighbourhood $φ \colon T^* L \dashrightarrow M$ containing $L'$.

  \pause

  \begin{definition}
    A \textbf{versal deformation} $v_L \colon H¹(L;ℝ) \dashrightarrow \symcal{L}$ is a local section of $\Flux \colon \symcal{L} \dashrightarrow H¹(L;ℝ)$
  \end{definition}

  \pause

  \begin{lemma}
    Let $v_L, v_L'$ be two versal deformations, then for $b ∈ H¹(L;ℝ)$ near zero $v_L(b) ≅ v_L'(b)$.
  \end{lemma}

  \pause

  \begin{lemma}
    Given $v_L$ and any invariant $\symcal{L} → I$ there is a well-defined germ $H¹(L;ℝ) \dashrightarrow I$ independent of $v_L$.
  \end{lemma}
\end{frame}

\begin{frame}{Displacement Energy Germ}
  Displacement Energy $e:\symcal{L} → ℝ$ is invariant under Hamiltonian isotopies.

  \begin{definition}[Displacement Energy Germ]
    The displacement energy germ $\symcal{E}_L \colon H¹(L;ℝ) \dashrightarrow ℝ$ gives a well-defined invariant.
  \end{definition}
  
  \pause

  \begin{example}
    In $ℂ^n$, $e(T(a_1,…,a_n)) = \min\{a_1,…,a_n\}$.
  \end{example}

  \begin{tikzpicture}
    \fill[opacity=0.1] (0,3) rectangle (3,0);
    \draw[thick] (0,3) -- (0,0) -- (3,0);
    \fill (1,1) circle[radius=1pt] node[anchor=west] {$T(1,1)$}
      (1,2) circle[radius=1pt] node[anchor=south] {$T(1,2)$};
    \foreach \x in {0.2,0.4,...,3}
      \draw[very thin] (\x,3) -- (\x,\x) -- (3,\x);
  \end{tikzpicture}
  \pause
  \hspace{1cm}
  \begin{tikzpicture}
    \draw (0,-1) node[anchor=north] {$\symcal{E}_{T(1,2)}$};
    \clip[draw] (0,0) circle[radius=1];
    \foreach \x in {-1,-0.8,...,1}
      \draw[very thin] (\x,1) -- (\x,-1);
    \fill (0,0) circle[radius=1pt];
  \end{tikzpicture}
  \hspace{1cm}
  \begin{tikzpicture}
    \draw (0,-1) node[anchor=north] {$\symcal{E}_{T(1,1)}$};
    \clip[draw] (0,0) circle[radius=1];
    \foreach \x in {-1,-0.8,...,1}
      \draw[very thin] (\x,1) -- (\x,\x) -- (1,\x);
    \fill (0,0) circle[radius=1pt];
  \end{tikzpicture}
\end{frame}

\begin{frame}[fragile]{Flux Map gives Hamiltonian Torus Actions}
  \begin{tikzcd}[ampersand replacement=\&]
    \begin{tikzpicture}[scale=1.5,baseline=0]
      \fill[blue,opacity=0.1] (0,1) rectangle (3,-1);
      \draw[thick] (0,1) -- (0,-1) (1,0) node[cross] {};
      \fill (0.5,0) circle [radius=1pt] node[anchor=south] {$T_{Ch}$};
      \draw[dashed] (1,0) .. controls ++(0.2,-.5) .. (1.2,-1);
    \end{tikzpicture}
    \arrow[r,"\Flux"] \&%
    \begin{tikzpicture}[scale=1.5,baseline=0]
      \fill[opacity=0.1] (0,1) -- (3,1) -- (3,-1) -- (1.4,-1) .. controls (1.3,-0.5) .. (1,0) .. controls (0.8,-0.5) .. (0.4,-1) -- (0,-1);
      \draw[thick] (0,1) -- (0,-1) (1,0) node[cross] {};
      \draw[dashed] (1.4,-1) .. controls (1.3,-0.5) .. (1,0) .. controls (0.8,-0.5) .. (0.4,-1);
      \fill (0.3,0) circle [radius=1pt] node[anchor=south] {$T_{Ch}$};
    \end{tikzpicture}
  \end{tikzcd}
\end{frame}
\begin{frame}[fragile]{Flux Map gives Hamiltonian Torus Actions}
  \begin{tikzcd}[ampersand replacement=\&]
    \begin{tikzpicture}[scale=1.5,baseline=0]
      \fill[blue,opacity=0.1] (0,1) rectangle (3,-1);
      \draw[thick] (0,1) -- (0,-1) (1,0) node[cross] {};
      \fill (0.5,0) circle [radius=1pt] node[anchor=south] {$T_{Ch}$};
      \draw[dashed] (1,0) -- (3,0);
    \end{tikzpicture}
    \arrow[r,"\Flux"] \&%
    \begin{tikzpicture}[scale=1.5,baseline=0]
      \fill[opacity=0.1] (0,1) rectangle (3,-1);
      \draw[thick] (0,1) -- (0,-1) (1,0) node[cross] {};
      \draw[dashed] (1,0) -- (3,0);
      \fill (0.3,0) circle [radius=1pt] node[anchor=south] {$T_{Ch}$};
    \end{tikzpicture}
  \end{tikzcd}
\end{frame}

\begin{frame}{Almost Toric Fibrations}
  \begin{minipage}{10cm}
    \begin{definition}[Almost Toric Fibration]
      Lagrangian fibration allowing for the following fibres:

      \begin{tblr}{cc}
        {regular\\\emph{2D stratum}} &
        \tikz[baseline=0] \draw (0,0) pic {torus}; & 
        \tikz[baseline=0,scale=.5] \fill[opacity=0.2] (0,1) rectangle (2,-1); \\
        \hline
        {elliptic-regular\\\emph{1D stratum}} &
        \tikz[baseline=0] \draw[thick] (0,0) circle [x radius=.75, y radius=0.35]; &
        \tikz[baseline=0,scale=.5]{\fill[opacity=0.2] (0,1) rectangle (2,-1); \draw[thick] (0,1) -- (0,-1);}\\
        \hline
        {elliptic-elliptic\\\emph{isolated}} & \tikz \fill (0,0) circle [radius=1pt]; &
        \tikz[baseline=0,scale=.5]{\fill[opacity=0.2] (0,1) rectangle (2,-1); \draw[thick] (0,1) -- (0,-1) -- (2,-1);}\\
        \hline
        {focus-focus\\\emph{isolated}} &
        \begin{tikzpicture}[baseline=0]
          \useasboundingbox (-.3,0.5) rectangle (1,-.5);
          \draw (1,0) .. controls (-1,1.5) and (-1,-1.5) .. (1,0)
            (1,0) .. controls ++(-.4,0.2) and ++(0.25,0.25) .. (-.03,0.0)
            (1,0) .. controls ++(-.4,-0.2) and ++(0.25,-0.25) .. (-.15,0.1);
        \end{tikzpicture} &
        \tikz[baseline=0,scale=.5]{\fill[opacity=0.2] (0,1) rectangle (2,-1); \draw[dashed] (1,0) node[cross] {} -- (2,0.5);}
      \end{tblr}
    \end{definition}
  \end{minipage}
\end{frame}

\begin{frame}{Almost Toric Fibrations}
  neighbourhood of focus-focus in action angle coordinates

  ATF diagrams, mutations
\end{frame}

\begin{frame}{Fun Facts about Almost Toric Fibrations}
  The following induce fibred symplectomorphisms outside a compact neighbourhood:
% Nodal Slide
\begin{tikzcd}[ampersand replacement=\&]
\begin{tikzpicture}[scale=0.5,use Hobby shortcut]
  \fill[opacity=0.1] (0,2) rectangle (4,-2);
  \fill[opacity=0.2] ([closed]1,-.2) .. (0.8,-.5) .. (1,-.7) .. (3,0.2) .. (3.2,0.5) .. (3,0.7);
  \draw[dashed] (0,-1) -- (1,-0.5) node[cross] {};

\end{tikzpicture} \rar{\text{nodal}}[swap]{\text{slide}} \&%
\begin{tikzpicture}[scale=0.5,use Hobby shortcut]
  \fill[opacity=0.1] (0,2) rectangle (4,-2);
  \fill[opacity=0.2] ([closed]1,-.2) .. (0.8,-.5) .. (1,-.7) .. (3,0.2) .. (3.2,0.5) .. (3,0.7);
  \draw[dashed] (0,-1) -- (3,0.5) node[cross] {};
\end{tikzpicture} \&
% Nodal trade
\begin{tikzpicture}[use Hobby shortcut,baseline=1cm]
  \draw[thick] (0,2) -- (0,0) -- (2,0);
  \clip (0,2) rectangle (2,0);
  \fill[opacity=0.1] (0,2) rectangle (2,0);

  \fill[opacity=0.2] ([closed]0,0.2) .. (1.1,1.2) .. (1.2,1.1) .. (0.2,0) .. (-.2,-.2);
\end{tikzpicture} \rar{\text{nodal}}[swap]{\text{trade}} \&%
\begin{tikzpicture}[use Hobby shortcut,baseline=1cm]
  \draw[thick] (0,2) -- (0,0) -- (2,0);
  \clip (0,2) rectangle (2,0);
  \fill[opacity=0.1] (0,2) rectangle (2,0);

  \draw[dashed] (0,0) -- (1,1) node[cross] {};

  \fill[opacity=0.2] ([closed]0,0.2) .. (1.1,1.2) .. (1.2,1.1) .. (0.2,0) .. (-.2,-.2);
\end{tikzpicture}
\end{tikzcd}
\end{frame}

\begin{frame}{Exoticity of the Chekanov torus}
  \begin{tikzcd}[ampersand replacement=\&]
    \begin{tikzpicture}
      \fill[opacity=0.1] (0,2) rectangle (2,0);
      \draw[thick] (0,2) -- (0,0) -- (2,0);
      \foreach \x in {0.1,0.2,...,2}
        \draw[very thin,gray] (\x,2) -- (\x,\x) -- (2,\x);
    \end{tikzpicture} \rar{\text{nodal}}[swap]{\text{trade}} \&
    \begin{tikzpicture}
      \fill[opacity=0.1] (0,2) rectangle (2,0);
      \foreach \x in {0.1,0.2,...,2}
        \draw[very thin,gray] (\x,2) -- (\x,\x) -- (2,\x);
      \draw[white,very thick] (0,0) -- (1,1);
      \draw[thick] (0,2) -- (0,0) -- (2,0);
      \draw[dashed] (0,0) -- (1,1) node[cross] {};
    \end{tikzpicture} \rar{ℤ²\text{-affine}}[swap]{\text{transform}} \&
    \begin{tikzpicture}
      \fill[opacity=0.1] (0,2) -- (0,0) -- (2,-2) -- (2,2);
      \foreach \x in {0.1,0.2,...,2}
        \draw[very thin,gray] (\x,2) -- (\x,0) -- (2,-2+\x);
      \draw[white,very thick] (0,0) -- (1,0);
      \draw[thick] (0,2) -- (0,0) -- (2,-2);
      \draw[dashed] (0,0) -- (1,0) node[cross] {};
    \end{tikzpicture} \rar{\text{mutation}} \&
    \begin{tikzpicture}
      \fill[opacity=0.1] (0,2) rectangle (2,-2);
      \foreach \x in {0.1,0.2,...,2}
        \draw[very thin,gray] (\x,2) -- (\x,-2);
      \draw[white,very thick] (0,0) -- (1,0);
      \draw[thick] (0,2) -- (0,-2);
      \draw[dashed] (1,0) node[cross] {} -- (2,0);
      \visible<2->{\fill (0.5,0) circle [radius=1pt] node[anchor=south] {$T_{Ch}$};}
    \end{tikzpicture}
  \end{tikzcd}

  \pause
  \pause

  \begin{tikzpicture}[scale=0.8]
    \draw (0,-1) node[anchor=north] {$\symcal{E}_{T(a_1,a_2)}$};
    \clip[draw] (0,0) circle[radius=1];
    \foreach \x in {-1,-0.8,...,1}
      \draw[very thin] (\x,1) -- (\x,-1);
    \fill (0,0) circle[radius=1pt];
  \end{tikzpicture}
  \hspace{1cm}
  \begin{tikzpicture}[scale=0.8]
    \draw (0,-1) node[anchor=north] {$\symcal{E}_{T(a,a)}$};
    \clip[draw] (0,0) circle[radius=1];
    \foreach \x in {-1,-0.8,...,1}
      \draw[very thin] (\x,1) -- (\x,\x) -- (1,\x);
    \fill (0,0) circle[radius=1pt];
  \end{tikzpicture}
  \hspace{1cm}
  \begin{tikzpicture}[scale=0.8]
    \draw (0,-1) node[anchor=north] {$\symcal{E}_{T_{Ch}(a)}$};
    \clip[draw] (0,0) circle[radius=1];
    \foreach \x in {-1,-0.8,...,1}
      \draw[very thin] (\x,1) -- (\x,1pt) (\x,-1pt) -- (\x,-1);
    \draw[very thick,opacity=0.3] (-1,0) -- (1,0);
    \fill (0,0) circle[radius=1pt];
  \end{tikzpicture}
\end{frame}

\section{\texorpdfstring{$B_{dpq}$}{Bdpq}}

\begin{frame}{$B_{dpq}$}
  \begin{minipage}{9cm}
    Let $P(z) = Π_{i=0}^d (z^p-a_i)$ for $0<a_1<…<a_d$ and $p ≥ 0$, and $q$ coprime to $p$.
    \[ M_P = \qty{z_1 z_2 - P(z_3) = 0} ⊂ ℂ^3\]
    \pause
    Let $ρ_p$ group of $p$-th roots of unity act on $ℂ³$ by
    \[
      ρ ⋅ (z_1,z_2,z_3) = (ρ z_1, ρ^{-1} z_2, ρ^q z_3)
    \]
    \pause
    $B_{dpq} = M_P/ρ_p$ with ATF $\symbf{H} \colon B_{dpq} → B$ given by
    \[
      \symbf{H}([z_1,z_2,z_3]) = (\abs{z_3}^{2}, \abs{z_1}²-\abs{z_2}²)
    \]
    \pause
    \vspace{-.9cm}
    \uncover<5->{\[\mu ≔ \Flux ∘ \symbf{H} \;\; \text{with this choice of branch cut}\]}
  \end{minipage}
  \hspace{1cm}
  \begin{tikzcd}[ampersand replacement=\&]
    \begin{tikzpicture}[scale=0.7]
      \fill[opacity=0.1,blue] (0,-1.5) rectangle (4,1.5);
      \draw[thick] (0,-1.5) -- (0,1.5);
      \draw (1,0) node[cross] {} (2,0) node[cross] {} (3,0) node[cross] {};
    \end{tikzpicture} \dar{\Flux} \\
    \begin{tikzpicture}[scale=0.7]
      \useasboundingbox (0,-1) rectangle (4,2);
      \fill[opacity=0.1] (0,-1) rectangle (4,2);
      \draw[thick] (0,-1) -- (0,2);
      \draw[dashed,->] (1,.5) node[cross] {} -- (2,1) node[cross] {} -- (3,1.5) node[cross] {} -- (4.2,2.1) node[anchor=north west] {$(p,q)$};
    \end{tikzpicture}
  \end{tikzcd}
\end{frame}

\begin{frame}{Lower Bound on Displacement Energy: J-holomorphic curves}
  $(X,ω)$ geometrically bounded symplectic manifold, $L ⊂ X$ a Lagrangian. And $J$ some almost complex structure.
  \begin{definition}
    \begin{align*}
      σ_S(X,J) &≔ \inf\{\text{Area}(J\text{-holomorphic sphere} ⊂ X)\}\\
      σ_D(X,L,J) &≔ \inf\{\text{Area}(J\text{-holomorphic disk} ⊂ (X,L))\}
    \end{align*}
  \end{definition}

  \pause
  \begin{theorem}[Chekanov]
    For every tame almost complex structure $J$, we have $e(L) ≥ \max\{σ_S(X,J), σ_D(X,L,J)\}$.
  \end{theorem}
\end{frame}

\begin{frame}{Lower Bound on Displacement Energy: J-holomorphic curves}
  $\symbf{H}_1 \colon B_{dpq} → ℝ$ is plurisubharmonic.\\
  \pause
  $⇒$ any $J$-disk $u \colon (D,∂D) → (B_{dpq},L)$ cannot have a maximum of $\symbf{H}_1 ∘ u$ in its interior.

  \pause
  \begin{tikzcd}[ampersand replacement=\&]
    \begin{tikzpicture}[scale=0.7]
      \fill[opacity=0.1,blue] (0,-1.5) rectangle (4,1.5);
      \draw[thick] (0,-1.5) -- (0,1.5);
      \fill (2,1) circle[radius=1pt] node[anchor=west] {$L$};
      \draw (1,0) node[cross] {} (2,0) node[cross] {} (3,0) node[cross] {};
    \end{tikzpicture} \rar{\text{nodal}}[swap]{\text{slide}} \&
    \begin{tikzpicture}[scale=0.7]
      \fill[opacity=0.1,blue] (0,-1.5) rectangle (4,1.5);
      \draw[thick] (0,-1.5) -- (0,1.5);
      \fill (2,1) circle[radius=1pt] node[anchor=west] {$L$};
      \draw (3,0) node[cross] {} (3.33,0) node[cross] {} (3.66,0) node[cross] {};
      \uncover<3->{\fill[green,opacity=0.3,text opacity=0.8] (0,-1.5) rectangle (2,1.5) (1,0) node[black] {$U$};}
    \end{tikzpicture}
  \end{tikzcd}
  \pause

  $u\colon (D, ∂D) → (U,L)$ \pause and $H_1(U,L) = ⟨D_0⟩ ≅ ℤ$, \pause so $∫_u ω ≥ ∫_{D_0} ω = μ_1(L)$.

  So $e(L) ≥ μ_1(L)$.
\end{frame}
\begin{frame}{Lower Bound on Displacement Energy: J-holomorphic curves}
  embedded version.
\end{frame}
\begin{frame}{Upper Bound on Displacement Energy: Probes}
  
\end{frame}

\begin{frame}{Displacement Engergy Germs Examples}
  
\end{frame}

\begin{frame}{A Fun Application}
  
\end{frame}


\end{document}

